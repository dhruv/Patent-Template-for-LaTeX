%%%%%%%%%%%%%%%%%%%%%%%%%%%%%%%%%%%%%%%%%%%%%%%%%%%%%%%%%%%%%%%%%%%%%%%%
%% -------------  Detailed description of the patent  --------------- %%
%%
%% NOTES:
%%  - Describe how the invention is made and how it works referencing
%%    the drawings.  Charts and tables can be included in the text of
%%    the application but are kept separate as drawings.
%%  - Start with a 'birdseye' paragraph that names the major
%%    components and a short phrase after each describing their
%%    functions.  Identifying those components with even reference
%%    numbers, starting with the number 10 to identify the preferred
%%    embodiment.
%%  - Number the parts chronologically (in 'drawings/features.def') 
%%  - Describe each of the major components in separate paragraphs
%%    after the birdseye paragraph.  Refer the reader to the drawing
%%    FIG. that best shows that component.
%%  - Typically, description of how the inventions works is found
%%    after the description of the structure (although embedding this
%%    information during the description of the structure might be
%%    easier for certain inventions).
%%  - The application as a whole must show how to MAKE and how to USE
%%    the invention.  If both of these are not clearly expressed, the
%%    Patent Office may reject the application.
%%  - Some examiners can be very 'dense' - it is best to add language
%%    pertaining to how, when, and where your invention is to be used.
%%
%% FORMATTING:
%%  \pa starts a new paragraph and numbers it sequentially             %%

\pa
\mbox{\ref{fig:assembly}} shows a typical prior art mechanical pencil assembly...

\pa
\mbox{\ref{fig:front}} shows...

\pa
\mbox{\ref{fig:cross-side-normal}} shows a perspective view of each part of the \pencil\ of the present invention, generally comprising an \eraser\ housed within an \erasercup, and a plurality of \pencilleads\ stored therein.

%% General closing paragraph - usually OK to leave this
\paragraph{Modifications}
It will be appreciated that still further embodiments of the present invention will be apparent
to those skilled in the art in view of the present disclosure.  It is to be understood that the
present invention is by no means limited to the particular constructions herein disclosed and/or
shown in the drawings, but also comprises any modifications or equivalents within the scope of the
invention.

